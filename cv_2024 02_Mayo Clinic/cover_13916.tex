\documentclass[10pt]{article}
\usepackage{fullpage}
\usepackage[left=0.75in,
			right=0.75in,
			top=0.75in,
			bottom=0.75in]{geometry}

\pagenumbering{gobble}

\begin{document}

\noindent To Whom It May Concern,\\

I am writing to express my interest in the Senior Portfolio Manager position in Labor Relations at REI. I assume that my background may be somewhat unusual at REI, but I am excited by the chance to bring my experience in economics as well as strategic analyses and development to help REI continue to understand and meet the needs of its customers, provide an environment for employees that values diversity and offers opportunities for growth across many dimensions, and continue to improve accessibility and inclusivity in the outdoors. 

I earned my BA in Math and Economics from Willamette University in 2012 and earned my Ph.D. in Economics from the University of Oregon in 2017. My studies were focused in Industrial Organization, which considers how to measure and quantify consumer preferences, how firms organize and interact within a market, and how the workforce structures itself and interacts with employers. My research heavily employed statistical analysis and structural modeling to understand interactions and decisions in historical data as well as forecast behaviors and events we'd expect to see in the future.

I joined Sandia National Laboratories (``Sandia'') in Albuquerque, NM in 2017. I began my career as a Cybersecurity Researcher, applying my background in statistics to estimate and assess the resilience of cyber-physical systems to natural and manmade disruptions. During this time, I also contributed to the launch risk analysis of the Mars 2020 mission, which landed the Perseverance rover on Mars in February 2021, and multiple economic impact analyses including an assessment of disruptions caused by COVID-19, which earned recognition from the Director of the Department of Energy Office of Science.

I joined Sandia's Strategic Futures and Policy Analysis group as a Systems Analyst in 2020. This group functions as an independent consulting arm for Sandia, providing objective policy analysis and decision support by drawing on a wide array of expertise and experiences from across and beyond Sandia. In this role, I have conducted and contributed to strategic studies that inform Sandia's strategy over the coming 10-30 years in areas such as knowledge management and workforce development. This includes organizing and leading workshops that engaged a range of stakeholders and subject matter experts to better understand and balance competing constraints and objectives. These insights directly contributed to workforce studies that help staff navigate COVID-19 and the ongoing transition to a hybrid workplace. These studies and efforts frequently involve helping stakeholders identify their objectives and needs, facilitating communication that improves understanding of and collaboration around competing objectives, and digesting as well as reconciling insights across many subject matter experts and other sources. I believe my background and experience in conducting these strategic studies and quantitative analyses will be directly applicable in this role within the Labor Relations team to ensure initiatives deliver intended outcomes in partnership with business stakeholders, HR team, and beyond.

Finally, living in New Mexico has really strengthened my love and appreciation of the outdoors. My wife and I are taking full advantage of the incredible public lands New Mexico has to offer, embracing hiking, backpacking, climbing, and mountaineering (of course with the help of expertise and gear from our local REI). Our experiences in the freedom of nature have amplified my passion for the outdoors and highlighted how important it is that everyone have access to amazing public resources like our national forests and parks. I hope to bring my background, skills, and experiences to help REI inspire passion for the outdoors and continue to improve accessibility and inclusivity.

\noindent \\Sincerely,\\\\
Taylor McKenzie

\end{document}