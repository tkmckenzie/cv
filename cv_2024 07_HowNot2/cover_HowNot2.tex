\documentclass[10pt]{article}
\usepackage{fullpage}
\usepackage[left=0.75in,
			right=0.75in,
			top=0.75in,
			bottom=0.75in]{geometry}

\pagenumbering{gobble}

\begin{document}
	
% ambiguity, business transformation/process improvement inititatives, cross-functional, 

\noindent Dear HowNot2 Team,\\

I'm really excited about your posting for Chief Nerd Officer and looking forward to the opportunity to talk with you all about what exciting things we can do together and in partnership with others.

In addition to the roles you have outlined, there are a few avenues I would be really excited to explore and I think would make further meaningful contributions to how climbers understand and make decisions around risk:
\begin{center}
\begin{tabular}{|p{0.5\linewidth}|p{0.5\linewidth}|}
\hline
\textbf{Technical effort} & \textbf{Impact} \\ \hline \hline
Analysis of low-probability/high-consequence system failures, especially focusing on describing tails of failure stress distributions using principles from uncertainty quantification and experimental design. & Most testing is focused on average failure stress, which tells us about what the ``average'' failure could look like. Focusing on tails of the failure stress distribution would tell us how frequently failures below critical thresholds (e.g., 5kN) could occur, which is a key component in understanding uncertainty and risk around these systems. \\ \hline
Analysis of failure modes and failure stress distributions to (1) support development of new standards (e.g., for rope abrasion/cutting), and (2) analyze safety and risk of using gear sold by HowNot2 for particular applications (e.g., gear for glacier travel). & HowNot2 has established itself as an authority in gear testing, especially for configurations outside of current standards and/or standard operating environments. \\ \hline
Analysis of how human factors combined with system configuration choices could significantly affect safety and risk. & \\ \hline
Consideration of trends (e.g., climate change, advancements in gear, changes in demographics/skills/preferences of climbers) that could affect safety of systems in the future & \\ \hline
Review of how academic disciplines and natural-hazard fields consider, evaluate, and mitigate risks. Development of educational materials to communicate these findings in the context of climbing, highlining, and other outdoor activities pursued by the HowNot2 team and audience. & \\ \hline
\end{tabular}
\end{center}

\begin{itemize}
	\item Analysis of low-probability/high-consequence system failures, especially focusing on describing tails of failure stress distributions using principles from uncertainty quantification and experimental design
	\item 
	\item Consideration of human
\end{itemize}

\noindent \\Sincerely,\\\\
Taylor McKenzie (he/him)\\
tkmckenzie@gmail.com\\
(509)430-0031

\end{document}