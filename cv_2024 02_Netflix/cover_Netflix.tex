\documentclass[10pt]{article}
\usepackage{fullpage}
\usepackage[left=0.75in,
			right=0.75in,
			top=0.75in,
			bottom=0.75in]{geometry}

\pagenumbering{gobble}

\begin{document}
	
% ambiguity, business transformation/process improvement inititatives, cross-functional, 

\noindent February 12, 2024\\\\
\noindent Dear Hiring Team,\\

I am very excited to be applying for the Data Engineer position at Netflix. Netflix's contributions to data analytics and decision support were a major inspiration for me pursuing education in economics and statistics and seeking work that leverages data analysis to inform intelligent decisions and strategies. I'm looking forward to the opportunity to join and work together with this talented and diverse team of data engineers to inform and enable the next stage of innovation at Netflix. 

In my current role in Sandia National Laboratories' Strategic Futures and Policy Analysis group, which serves as an independent consulting arm of the laboratories, I work together with executives and other stakeholders to help develop strategy and provide objective assessments of their systems and plans of action. As a part of this work, I have designed and conducted strategic foresight studies that investigate how Sandia could be better prepared to respond to economic, technological, and geopolitical trends and uncertainties over the coming decades. Given the wide range of what the future may hold, these studies are necessarily ambiguous at their inception. Using established foresight and expert elicitation techniques, we engage with multidisciplinary and cross-functional groups of stakeholders and subject matter experts to identify foreseeable challenges and disruptive dynamics. We draw on trusted relationships across the laboratory, built through consistent engagement and rigorous objectivity, to consider the confluence of multiple dynamics and identify second- and higher-order effects that could very consequential but are often less obvious to individual stakeholders. We synthesize and distill our findings to produce valuable insights for executives and actionable plans for stakeholders, enabling Sandia to be better prepared for an increasingly dynamic and uncertain future. I then work with stakeholders through implementation to gather and analyze metrics of initiative success and identify adjustments that can improve implementation or more effectively mitigate risks. I have personally conducted and contributed to a number of these foresight studies and policy efforts, including informing Sandia's transition to remote and hybrid work through COVID-19, developing laboratory strategy around the future of knowledge management and workforce development, and guiding facilities and infrastructure investment plans in the face of rapidly-evolving economic and geopolitical environments.

Throughout my time at Sandia, I have also conducted objective risk and reliability analyses for a number of critical national security systems. As an example, I serve on the Interagency Nuclear Safety Review Board (INSRB), which conducts independent assessments of safety analyses of spacecraft that carry significant amounts of nuclear material onboard, such as the two most recent missions that landed rovers on Mars. Ultimately, these analyses are delivered to and signed off by the President of the United States, acknowledging that risk of radiation exposure to the public is acceptable for the given mission. As a result, it is critical that these safety analyses both accurately reflect our best understanding of risk to the public as well as incorporate and convey all relevant sources of uncertainty. Through my role on INSRB, I have leveraged my expertise and passion for statistics and uncertainty quantification and drawn on my technical background in Bayesian statistics and simulation to continuously push for improvements to our methodology and our understanding of risk. These efforts, while very important, are frequently challenging and professionally risky, requiring questioning, carefully assessing, and, where appropriate, overturning norms and methods that have been held across decades of these analyses. By drawing on the expertise of a multi-disciplinary and cross-functional team and thoughtfully engaging with sponsors and stakeholders, we improved our understanding of risk for the launch of the Mars 2020 Perseverance Rover and improved the process to evaluate risk for future missions. My experiences both on INSRB and as a cybersecurity researcher in Sandia's Cyber Resilience group, where I worked together with engineers and leveraged my background in simulation to quantify resilience of cyber-physical industrial control systems, have strengthened my enjoyment and appreciation of working with diverse multi-disciplinary teams.

I believe my experiences at Sandia exemplify and have strengthened values emphasized by Netflix's culture. My passion for excellence and intellectual curiosity have yielded innovative approaches to challenging problems that have provided decision-makers with improved insights and understanding to guide their decisions. I have relied on teams that are diverse across many dimensions and bring many different valuable experiences to the problem at hand, resulting in analysis products and insights that are robust and well-founded as well as appealing to a broad audience. I am not afraid to challenge existing norms and approaches, and I seek to understand stakeholder objectives and constraints to make informed process improvements. I am excited by the opportunity to leverage these skills and collaborate with others who also embody these values to continue and further Netflix's legacy of innovation.


\noindent \\Sincerely,\\\\
Taylor McKenzie

\end{document}