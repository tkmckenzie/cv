\documentclass{beamer}

\usepackage{multirow}

\usepackage{tikz}
\usetikzlibrary{shapes,arrows}

\newcommand{\ep}{\varepsilon}
\newcommand{\img}[3]{\begin{figure}\centering\includegraphics[width=#3in]{figures/#1}\caption{#2}\end{figure}}

\begin{document}
	
\tikzstyle{block} = [rectangle, draw, fill=blue!20, 
text width=5em, text centered, rounded corners, minimum height=4em]
\tikzstyle{line} = [draw, -latex']
	
%%%%% Background & Work at Sandia %%%%%
\begin{frame}{Motivating Example: Description}
\begin{itemize}
	\item Let's begin with a simple example:
		\begin{itemize}
			\item Construct a model of outputs $y$ based on inputs $X$
			\item Interested in estimating/describing $y|X$ and quantifying uncertainty in that estimate
		\end{itemize}
	\item These data are based on actual experiments performed on cyber system
		\begin{itemize}
			\item $X$: Bandwidth of servers on the network
			\item $y$: Throughput of data across the network
		\end{itemize}
\end{itemize}
\end{frame}

\begin{frame}{Motivating Example: Sample Data}
	\img{lm-error-dist-standard-point.pdf}{Sample Data}{4}
\end{frame}

\begin{frame}{Motivating Example: Standard Approach}
\begin{itemize}
	\item Standard approach:
		\begin{enumerate}
			\item Relationship between $X$ and $y$ looks linear
			\item Perform ordinary least squares (OLS) regression
			\item (Possibly) Examine diagnostics to check validity of assumptions
		\end{enumerate}
\end{itemize}
\end{frame}

\begin{frame}{Motivating Example: Linear Regression}
	\img{lm-error-dist-standard-fit.pdf}{Fitted Sample Data}{4}
\end{frame}

\begin{frame}{Motivating Example: Diagnostics}
	\img{lm-error-dist-standard-fitted-values.pdf}{Residuals vs. Fitted Values}{4}
\end{frame}

\begin{frame}{Motivating Example: Diagnostics}
	\img{lm-error-dist-standard-qq.pdf}{Normal Quantile Plot}{4}
\end{frame}

\begin{frame}{Motivating Example: Standard Practice}
\begin{itemize}
	\item Many practitioners may not have run diagnostics
	\item If diagnostics were run, many practitioners will
		\begin{itemize}
			\item use residuals vs. fitted values as evidence that errors are uncorrelated and homoskedastic (constant variance).
			\item use quantile plot as evidence that errors are approximately normal, with exception of one outlier.
		\end{itemize}
	\item An especially diligent practitioner may run test of normality of residuals
		\begin{itemize}
			\item Shapiro-Wilk test $p$-value: 0.03917
		\end{itemize}
\end{itemize}
\end{frame}

\begin{frame}{Gauss-Markov Theorem}
\begin{itemize}
	\item However, most practitioners won't worry about normality
	\item Gauss-Markov theorem states OLS estimate is best linear unbiased estimate as long as
	\begin{itemize}
		\item errors have mean zero
		\item errors have constant finite variance
		\item errors are uncorrelated
	\end{itemize}
	\item While normality is needed for inference (e.g., significance, confidence intervals), it is not needed to produce unbiased parameter estimates or predictions
\end{itemize}
\end{frame}

\begin{frame}{Finite Variance Assumption}
\begin{itemize}
	\item Gauss-Markov theorem requires that errors have constant \textbf{finite} variance
	\item Errors in the sample data are Cauchy distributed
		\begin{itemize}
			\item Cauchy and normal distributions look \textit{similar}
			\item Cauchy distribution has fat tails, to the point where the mean and variance do not exist
		\end{itemize}
\end{itemize}
\end{frame}

\begin{frame}{Normal and Cauchy Distributions}
	\img{normal-cauchy.pdf}{Normal and Cauchy Distributions}{4}
\end{frame}

\begin{frame}{Infinite Variance Consequences}
\begin{itemize}
	\item When errors are Cauchy distributed
		\begin{itemize}
			\item $E[y|X]$ does not exist, so cannot be estimated
			\item Weak law of large numbers does not hold, so OLS parameter estimates do not converge to true values (or any other value) as sample size increases
		\end{itemize}
\end{itemize}
\end{frame}

\begin{frame}{Modeling Process}
\resizebox{\columnwidth}{!}{
\begin{tikzpicture}[node distance = 3cm, auto]
	% Nodes
	\node [block] (question) {Question};
	\node [block, right of=question] (hypothesis) {Hypothesis};
	\node [block, right of=hypothesis] (theory) {Theoretical Model};
	\node [block, right of=theory] (empirics) {Empirical Model};
	\node [block, above of=empirics] (data) {Data};
	\node [block, right of=empirics] (evaluation) {Evaluation/\\diagnostics};
	\node [block, right of=evaluation] (results) {Results};
	% Edges
	\path [line] (question) -- (hypothesis);
	\path [line] (hypothesis) -- (theory);
	\path [line] (theory) -- (empirics);
	\path [line] (data) -- (empirics);
	\path [line] (empirics) -- (evaluation);
	\path [line] (evaluation.south) |- ++(0,-5mm) -| (empirics.south);
	\path [line] (evaluation.south) |- ++(0,-5mm) -| (theory.south);
	\path [line] (evaluation) -- (results);
\end{tikzpicture}
}
\end{frame}

\begin{frame}{Replication Crisis}
\begin{itemize}
	\item In academic publishing, studies are reviewed for logical consistency and to ensure best practices are followed
	\item Peer review should enhance reproducibility of studies
	\item However, many published findings have proven difficult to replicate
		\begin{itemize}
			\item Reviewers don't have time/resources to verify all details in theory/empirics
			\item Often completely impossible to verify assumptions of empirical model (access to data/code, infeasible to re-run analysis)
			\item Reviewers rarely see iterations on theory/empirics
		\end{itemize}
\end{itemize}
\end{frame}

\begin{frame}{Review of OUO/Classified Studies}
\begin{itemize}
	\item OUO and classified studies are rarely undergo any in-depth review
	\item When reviews are conducted, it can be especially difficult to obtain data and code used for analysis
\end{itemize}
\end{frame}

\begin{frame}{}
\begin{itemize}
	\item 
\end{itemize}
\end{frame}

\end{document}
