\documentclass[10pt]{article}
\usepackage{fullpage}
\usepackage[left=0.75in,
			right=0.75in,
			top=0.75in,
			bottom=0.75in]{geometry}

\pagenumbering{gobble}

\begin{document}

\noindent To Whom It May Concern,\\

I am writing to express my strong interest in the Information Scientist position at RAND. I hope to leverage my experience in analyzing risk and resilience of cyber-physical critical infrastructure systems, engaging with stakeholders to understand objectives and communicating results that inform their decisions, and conducting analyses that draws on a range of disciplines, methods, and data to provide decision-makers with objective and insightful analyses that will inform cyber strategy and policy into the future. I am excited to collaborate with and learn from experts across RAND to better understand challenges to securing cyber-physical infrastructure and designing analyses that enable informed policy.

I received my Ph.D. in Economics from the University of Oregon in 2017. My thesis focused on adapting existing econometric methods of production and efficiency to be more flexible to provide detailed estimates describing technological change, productive capabilities, and market power. I applied these statistical methods to the U.S. freight rail industry to provide insights into changes that occurred in the industry in the period following its deregulation. Through these analyses, I showed how deregulation differentially affected railroads and illustrated the complexity of impacts for firms and consumers. 

I began working at Sandia National Laboratories as a Cybersecurity Researcher in 2017. Early in this role, I took formal training covering fundamentals of cyber systems and their vulnerabilities as well as contributed to a breadth of projects that provided me extensive familiarity with risks posed to cyber-physical systems. I leveraged my expertise in statistics and simulation to assess the risk and resilience of these systems and develop robust metrics to quantify system health utilizing Sandia's state-of-the-art emulation capabilities. During this time, I worked with US Army NETCOM to develop real-time measures of resilience of their networks, explored viability of and defenses against steganography over industrial control system communications, and investigated statistical methods to perform inference on cyber experiment outcomes that provide insights into system risk and resilience.

During this time, I became a member of the Interagency Nuclear Safety Review Board, which performs safety analyses of launches of spacecraft containing nuclear material. I was involved in the risk analysis of Perseverance and its multi-mission radioisotope thermoelectric generator, which launched July 2020 and landed on Mars in February 2021. During this effort, I coordinated with experts across a wide range of disciplines and with NASA stakeholders to understand and identify analysis needs and drew on my statistics expertise to implement cutting-edge uncertainty quantification methods into the risk assessment framework that improved understanding of mission risk for contemporary and future missions. My leadership in this analysis earned a Technical Excellence Award from Sandia in 2020.

I joined Sandia's Systems Analysis group in March 2020. I began by contributing to studies informing how the lab navigated the coming years of the COVID-19 pandemic, leading studies that assessed impact of COVID-19 and remote work to the workforce and identifying mitigations, and performing state- and national-level economic impact analyses as a part of Sandia's COVID-19 Pandemic Modeling Effort, for which I earned personal recognition from J. Stephen Brinkley, Acting Director of the DOE Office of Science. I have continued to inform Sandia's transition to a hybrid work environment, both enabling staff to work more effectively in this new paradigm as well as providing Sandia with more flexible employment options that improve its competitiveness in the labor market.

Most recently, I have taken part in Sandia's Global Futures study series, which explores trends and disruptions around national security topics, informing and inspiring discussions amongst Sandia's leadership to identify actions we could be taking today to put us on a more preferable trajectory into the future. I was involved in studies on Space Futures and the Future of Collective Security, and most recently led a study on the Future of Economic Value and National Security, which was briefed to Sandia leadership and stakeholders at the US Secret Service, the Office of Science and Technology Policy, and the National Institute for Standards and Technology. Each of these studies involved developing a technical understanding of the topic and Sandia's capabilities, engaging with experts across and beyond the lab, creatively and rigorously developing a perspective on future disruptions and trends, and informing a range of stakeholders on implications relevant to them to identify partnerships and collaborations.

I am excited to join the breadth of experts across RAND to enable decision-makers and stakeholders to make informed cyber strategy and policy. I hope to leverage my technical expertise in economics and statistics, understanding of cyber-physical systems and modeling, and experience using strategic foresight to inform policy discussions to provide objective insights into challenging and urgent national security issues. I would value the opportunity to learn more about the range of expertise at RAND and where my skills and experiences could be complementary and valuable in informing policy.

\noindent \\Sincerely,\\\\
Taylor McKenzie

\end{document}