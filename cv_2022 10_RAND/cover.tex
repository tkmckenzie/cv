\documentclass[11pt]{article}
\usepackage{fullpage}

\begin{document}

\noindent To Whom It May Concern,\\

\noindent I am writing to express my strong interest in the Information Scientist position at RAND. I 

I received my Ph.D. in Economics from the University of Oregon in 2017. My thesis focused on adapting existing econometric methods of production and efficiency to be more flexible to provide detailed estimates describing technological change, productive capabilities, and market power. I applied these statistical methods to the U.S. freight rail industry to provide insights into changes that occurred in the industry in the period following its deregulation. Through these analyses, I showed how deregulation differentially affected railroads and illustrated the complexity of impacts for firms and consumers. 

I began working at Sandia National Laboratories as a Cybersecurity Researcher in 2017. Early in this role, I was fortunate to be able to engage in training covering cyber systems and vulnerabilities as well as engage in a breadth of projects that gave me familiarity with cyberphysical systems and their vulnerabilities. I leveraged my expertise in statistics and simulation to assess the risk and resilience of these systems and develop robust metrics to quantify system health utilizing Sandia's state-of-the-art emulation capabilities. During this time, I worked with US Army NETCOM to develop real-time measures of resilience of their networks, explored viability of and defenses against steganography over industrial control system communications, and investigated statistical methods to perform inference on cyber experiment outcomes.

During this time, I also contributed to a number of other efforts across the lab. I joined the Interagency Nuclear Safety Review Board, which performs safety analyses of launches of spacecraft containing nuclear material. I was involved in the launch analysis of Perseverance and its multi-mission radioisotope thermoelectric generator, which launched July 2020 and landed on Mars in February 2021. During this effort, I coordinated with subject-matter experts within and beyond Sandia and with NASA stakeholders and drew on my statistics expertise to implement cutting-edge uncertainty quantification methods into the risk assessment framework that improved understanding of mission risk for contemporary and future missions.

I joined Sandia's Systems Analysis group in early 2020. The second week into my new position was marked by the transition to remote work due to the COVID-19 pandemic. I leveraged my experience of beginning with a new group in a fully remote work posture to help conduct and inform a series of COVID studies. I took part in a ``Telecommuting Best Practices'' study that helped transition Sandia's workforce into the over two years of remote work that would follow. I participated in a COVID Futures exercise that explored uncertainties around the pandemic presented a range of potential future scenarios to Sandia's leadership, helping them prepare for eventualities from multiple waves of the pandemic to tragic deaths of coworkers. Finally, I contributed to the COVID-19 Pandemic Modeling Effort by performing economic impact analyses of pandemic mitigation scenarios, which earned personal recognition from J. Stephen Brinkley, Acting Director of the DOE Office of Science.

I have been continuing 

\end{document}