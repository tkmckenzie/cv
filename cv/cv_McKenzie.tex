\documentclass[11pt]{article}
%\usepackage{fullpage}
%\usepackage{amsmath}
%\usepackage{amssymb}
%\usepackage[usenames]{color}
%\usepackage{multicol}

\leftmargin=0.25in
\oddsidemargin=0.25in
\textwidth=6.0in
\topmargin=-1in
\textheight=10in
\footskip=-10pt

\raggedright
\thispagestyle{empty}

\pagenumbering{gobble}

\def\bull{\vrule height 0.8ex width .7ex depth -.1ex }
% DEFINITIONS FOR RESUME

\newenvironment{changemargin}[2]{%
  \begin{list}{}{%
    \setlength{\topsep}{0pt}%
    \setlength{\leftmargin}{#1}%
    \setlength{\rightmargin}{#2}%
    \setlength{\listparindent}{\parindent}%
    \setlength{\itemindent}{\parindent}%
    \setlength{\parsep}{\parskip}%
  }%
  \item[]}{\end{list}
}

\newcommand{\lineover}{
	\begin{changemargin}{-0.05in}{-0.05in}
		\vspace*{-8pt}
		\hrulefill \\
		\vspace*{-2pt}
	\end{changemargin}
}

\newcommand{\header}[1]{
	\begin{changemargin}{-0.5in}{-0.5in}
		\scshape{#1}\\
  	\lineover
	\end{changemargin}
}

\newcommand{\contact}[4]{
	\begin{changemargin}{-0.5in}{-0.5in}
		\begin{center}
			{\Large \scshape {#1}}\\ \smallskip
			{#2}\\ \smallskip 
			{#3}\\ \smallskip
			{#4}\smallskip
		\end{center}
	\end{changemargin}
}

\newenvironment{body} {
	\vspace*{-16pt}
	\begin{changemargin}{-0.25in}{-0.5in}
  }	
	{\end{changemargin}
}	

\newcommand{\school}[4]{
	\textbf{#1} \hfill \emph{#2\\}
	#3\\ 
	#4\\
}

\newenvironment{blockquote}{%
	\par%
	\medskip
	\leftskip=1em\rightskip=1em%
	\noindent\ignorespaces}{%
	\par\medskip}

% END RESUME DEFINITIONS

\begin{document}

%%%%%%%%%%%%%%%%%%%%%%%%%%%%%%%%%%%%%%%%%%%%%%%%%%%%%%%%%%%%%%%%%%%%%%%%%%%%%%%%
% Name



\contact{\vspace*{-5ex}Taylor K McKenzie}{U.S. Citizen, DOE Q Clearance (DOD TS Equivalent)\\ \textbf{Address:} 1017 Adams St SE, Albuquerque, NM 87108}{\vspace*{-0.75ex}\textbf{E-mail:} tkmckenzie@gmail.com / \textbf{Cell Phone:} +1 (509) 430-0031}

%%%%%%%%%%%%%%%%%%%%%%%%%%%%%%%%%%%%%%%%%%%%%%%%%%%%%%%%%%%%%%%%%%%%%%%%%%%%%%%%
% Education
\vspace{-12pt}
\header{Education}

\begin{body}
	\vspace{14pt}
	\textbf{Ph.D. Economics}{} \hfill \emph{June 2017}{} \\
	\emph{University of Oregon}, Eugene, OR{}\\
	\emph{Dissertation: Railroads, Regulation, Efficiency, and Competition}\\
	\emph{Committee: Wesley W. Wilson, Van Kolpin, Jeremy Piger,}\\
	\ \ \ \ \emph{Diane Del Guercio, Keaton Miller}\\
	\vspace*{0pt}
  \medskip
	\textbf{M.S. Economics}{} \hfill \emph{December 2013}{} \\
	\emph{University of Oregon}, Eugene, OR{}\\
	\emph{Advisor: Dr. Wesley W. Wilson} \\
	\vspace*{0pt}
  \medskip
  	\textbf{B.A. Economics and Mathematics}{} \hfill \emph{May 2012}{} \\
	\emph{Willamette University}, Salem, OR{}\hfill \emph{Summa Cum Laude}{}\\
	\emph{Advisors: Dr. Raechelle Mascarenhas and Dr. Peter Otto} {} \hfill\\
\end{body}
\smallskip
%%%%%%%%%%%%%%%%%%%%%%%%%%%%%%%%%%%%%%%%%%%%%%%%%%%%%%%%%%%%%%%%%%%%%%%%%%%%%%%%
% Skills
% Experience
\header{Work Experience}

\begin{body}
	\vspace{14pt}
	
	\textbf{Sandia National Laboratories} \hfill \emph{Aug. 2017-Present}\\
	\emph{Senior Cybersecurity Researcher}\\
	\vspace*{-4pt}
	\begin{itemize}
		\item Extended economic and statistical expertise to resilience of cyber and cyber-physical systems.
		\item Participated in Information Design Assurance Red Teaming (IDART), including performing cybersecurity assessments.
		\item Applied knowledge of classical and Bayesian statistics to wide breadth of problems in risk analysis, drawing on diverse subject-matter expertise to produce well-founded statistical models.
		\item Contributed to economic analyses of resilience of communities and markets to natural and man-made disasters.
	\end{itemize}	
	
%	\vspace{14pt}
	
	\textbf{University of Oregon} \hfill \emph{Fall-Spring 2012-2017}\\
	\emph{Graduate Teaching Fellow}\\
	\emph{Department of Economics}
	\vspace*{-4pt}
	\begin{itemize}
		\item Served as teaching assistant to both undergraduate and graduate courses.
		\item Acted as independent instructor of five courses covering intermediate microeconomic theory, industrial organization, and development economics.
	\end{itemize}	
	
%	\vspace{14pt}

	\textbf{Pacific Northwest National Laboratory}  \hfill \emph{Summers of 2010 - 2012, 2014}\\
	\emph{National Security Intern}
	\hfill \emph{902 Battelle Blvd}\\ 
	\emph{Knowledge Discovery and Informatics Group}
	\hfill \emph{Richland, WA 99354}\\
	\emph{Mentors: Dr. Courtney Corley and Dr. Satish Chikkagoudar}
	\vspace*{-4pt}
	\begin{itemize} \itemsep -0pt  % reduce space between items
\item Developed a game-theoretic model of inter-organization email traffic and estimated the\\ model and produced simulations using Bayesian methods. Simulations were used to\\ identify risk of cybersecurity threats and develop strategies to mitigate the damage of\\ cyberattacks.
\item Developed methods to predict and interpret trends in social media and implemented those\\ methods in Python and R on PNNL's supercomputing cluster. Also co-authored ``SociAL\\ Sensor Analytics: Measuring Phenomenology at Scale.''
\item Developed a predictive disease model describing spread of SARS and cholera worldwide and prototyped both models in Python. Also developed estimates for air travel between countries and the effects of various intervention techniques such as airport screening and quarantine.
\item Developed metrics to analyze the effectiveness of biosurveillance systems and models and\\ investigation of economic indicators in nuclear proliferation pathway analysis.
	\end{itemize}

				
\end{body}
\vspace{-9pt}
\smallskip
%%%%%%%%%%%%%%%%%%%%%%%%%%%%%%%%%%%%%%%%%%%%%%%%%%%%%%%%%%%%%%%%%%%%%%%%%%%%%%%%

\pagebreak

%\header{Teaching Experience}
%
%\begin{body}
%	
%	\item \emph{Independent Instructor:}
%		\begin{itemize}
%			\item EC 311: Intermediate Microeconomic Theory \hfill \emph{Fall 2014}
%			\item EC 360: Issues in Industrial Organization \hfill \emph{Fall 2016, Spring 2017}
%			\item EC 390: Problems and Issues in Developing Economies \hfill \emph{Fall 2015, Spring 2016}
%		\end{itemize}
%	\item \emph{Teaching Assistant:}
%	\begin{itemize}
%		\item EC 201: Introduction to Microeconomic Analysis \hfill \emph{Fall 2012, Spring 2013, Spring 2015}
%		\item EC 202: Introduction to Macroeconomic Analysis \hfill \emph{Winter 2013}
%		\item EC 320: Introduction to Econometrics \hfill \emph{Fall 2013, Winter 2015}
%		\item EC 428: Behavioral and Experimental Economics \hfill \emph{Spring 2014}
%		\item EC 430: Urban and Regional Economics \hfill \emph{Spring 2014}
%		\item EC 607: Seminar: Core Macro \hfill \emph{Winter 2014}
%	\end{itemize}	
%	
%\end{body}
%\vspace{-9pt}
%\smallskip
%%%%%%%%%%%%%%%%%%%%%%%%%%%%%%%%%%%%%%%%%%%%%%%%%%%%%%%%%%%%%%%%%%%%%%%%%%%%%%%%

\header{Publications}

\begin{body}
\vspace{14pt}
\begin{itemize}
	\item F\"{a}re, Rolf, Taylor McKenzie, Wesley Wilson, and Wenfeng Yang (2017). Mergers, efficiency, and productivity in the railroad industry: An attribute-incorporated data envelopment analysis approach. Transportation Policy and Economic Regulation: Essays in Honor of Theodore Keeler.
	\item Corley, C.D., C. Dowling, S.J. Rose, and T. McKenzie (2013). SociAL Sensor Analytics: Measuring Phenomenology at Scale. \textit{2013 IEEE International Conference on Intelligence and Security Informatics}, 61-66.
	\item Corley, C.D., et al, including T. McKenzie (2012). Assessing the Continuum of Event-Based Biosurveillance Through an Operational Lens. \textit{Biosecurity and Bioterrorism 10(1)}, 131-41.
\end{itemize}
\end{body}

\smallskip
\header{Working Papers}

\begin{body}
\vspace{14pt}
\begin{itemize}
	\item McKenzie, Taylor. Markups and Scale Elasticities for Differentiated Railroad Networks (with Wesley W. Wilson).
	\item McKenzie, Taylor. Decomposing Changes in Productivity Using Bayesian Methods.
	\item McKenzie, Taylor. General Bayesian Marginal Likelihood Estimation Using Iterative Density Estimation.
\end{itemize}
\end{body}

\smallskip
\header{Works in Progress}

\begin{body}
	\vspace{14pt}
	\begin{itemize}
		\item McKenzie, Taylor. Estimation of Allocative Inefficiency Using Smooth Non-Parametric Frontier Analysis.
	\end{itemize}
\end{body}

\smallskip
\header{Developed Software}
\begin{body}
	\vspace{14pt}
	\begin{itemize}
		\item R package: Smooth Non-Parametric Frontier Analysis.
	\end{itemize}
\end{body}


\smallskip
\header{Skills}

\begin{body}
\vspace{14pt}
\begin{itemize}
	\item Development and implementation of classical and Bayesian statistical and econometric models in R, Python, Matlab, and Stata.
	\item Formal training and practical experience in cybersecurity, ranging from studies of mission and capability resilience of cyber-dependent systems to investigating specific cyber vulnerabilities.
	\item Experience working with diverse teams and developing models that synthesize theories and results from a multitude of disciplines.
\end{itemize}
\end{body}

\smallskip
\header{Awards and Recognitions}

\begin{body}
	\vspace{14pt}
	\begin{itemize}
		\item Best Dissertation Award from American Economic Association's\hfill \emph{Dec. 2017}\\ Transportation and Public Utilities Group.
		\item Ph.D. Research Paper Award from the University of Oregon for\hfill \emph{May 2016}\\ ``Markups and Scale Elasticities for Differentiated Railroad Networks.''
		\item Achievement Award from Pacific Northwest National Laboratory for work\hfill \emph{Aug. 2014}\\ on empirical game-theoretic modeling. 
		\item Best Paper Award at Institute of Electrical and Electronics Engineers Intelligence \hfill \emph{June 2013}\\ and Security Informatics Conference for ``SociAL Sensor Analytics: Measuring\\ Phenomenology at Scale.''
		\item Best First-Year Econometrics Performance Award from the University of Oregon. \hfill \emph{June 2013}
		\item National Security Directorate Outstanding Performance Award from Pacific\hfill \emph{Sept. 2011}\\ Northwest National Laboratory for work on predictive disease modeling. 
		\item The Chester F. Luther Mathematics Scholarship from Willamette University. \hfill \emph{May 2011}
		\item Phi Beta Kappa Member, Junior Inductee. \hfill \emph{May 2011}
	\end{itemize}
\end{body}
\smallskip

%\smallskip
%\header{Service}
%\begin{body}
%	\vspace{14pt}
%	\begin{itemize}
%		\item Referee for Economic Inquiry
%		\item TERF Room (computer lab for UO Economics Department) coordinator
%	\end{itemize}
%\end{body}

%\smallskip
%\header{Presentations}
%
%\begin{body}
%	\vspace{14pt}
%	\begin{itemize}
%		\item ``Decomposing Changes in Productivity Using Bayesian Methods.'' Western\hfill \emph{July 2016}\\ Economic Association International Conference.
%		\item ``Markups and Scale Elasticities for Differentiated Railroad Networks.'' Western\hfill \emph{July 2015}\\ Economic Association International Conference. 
%		\item ``Simulating Professional Communication with Economic Network Models.'' Pacific\hfill \emph{Aug. 2014}\\ Northwest National Laboratory. 
%		\item ``Social Media Analysis.'' Pacific Northwest National Laboratory. \hfill \emph{July 2012}
%		\item ``Air Travel and Infectious Diseases.'' Pacific Northwest National Laboratory. \hfill \emph{Aug. 2011}
%		\item ``Bayesian Networks in Disease Modeling.'' Willamette University Mathematics\hfill \emph{May 2011}\\ Colloquium.
%		\item ``Biosurveillance and FCAT.'' Pacific Northwest National Laboratory. \hfill \emph{Aug. 2010}
%	\end{itemize}
%\end{body}

%\smallskip
%\header{References}
%
%\begin{body}
%	\vspace{14pt}
%		\begin{tabular}{lcl}
%			\textbf{Wesley W. Wilson (Chair)} & \hspace{1in} & \textbf{Van Kolpin}\\
%			Professor & \hspace{1in} & Professor \\
%			University of Oregon & \hspace{1in} & University of Oregon\\
%			Department of Economics & \hspace{1in} & Department of Economics\\
%			(541) 346-4690 & \hspace{1in} & (541) 346-3011\\
%			wwilson@uoregon.edu & \hspace{1in} & vkolpin@uoregon.edu\\\\
%			\textbf{Jeremy Piger} & \hspace{1in} & \textbf{Keaton Miller}\\
%			Professor & \hspace{1in} & Assistant Professor \\
%			University of Oregon & \hspace{1in} & University of Oregon\\
%			Department of Economics & \hspace{1in} & Department of Economics\\
%			(541) 346-6075 & \hspace{1in} & (541) 346-4653\\
%			jpiger@uoregon.edu & \hspace{1in} & keatonm@uoregon.edu\\\\
%			\textbf{Diane Del Guercio} & \hspace{1in} & \textbf{Alfredo Burlando}\\
%			Professor & \hspace{1in} & Assistant Professor\\
%			University of Oregon & \hspace{1in} & Graduate Placement Director\\
%			Department of Finance & \hspace{1in} & University of Oregon\\
%			(541) 346-5179 & \hspace{1in} & Department of Economics\\
%			dianedg@uoregon.edu & \hspace{1in} & (541) 346-1351\\
%			 & \hspace{1in} & burlando@uoregon.edu
%		\end{tabular}
%\end{body}

%\pagebreak
%
%\smallskip
%\header{Working Paper Abstracts}
%
%\begin{body}
%	\vspace{14pt}
%	\emph{Technological Change and Productivity in the Rail Industry: A Bayesian Approach (Job Market Paper)}\\
%	\begin{blockquote}
%	Productivity and its growth are central to the long-term growth, and long-term viability of firms and industries. Partial deregulation of railroads was led by concerns that existing regulation and changes to the industry led to stagnation in productivity. Policy changes made it easier for firms to increase productivity through broad organizational changes like mergers and abandoning unprofitable routes as well as specific technological innovation through the 1980s and early 1990s. However, as the industry has become increasingly consolidated and as more lines have been abandoned, firms may need to rely on technological change to increase productivity. I develop and estimate a model that separates changes in productivity due to innovation and those caused by non-innovative factors and use Bayesian estimation. This allows productivity and technology to evolve flexibly across firms and through time, allowing an examination of changes in railroad productivity and identification of its driving component. I find that every Class I railroad has experienced growth in productivity since 1999. Improvements in technology were the driving factor in the growth of BNSF, KCS, Soo Line, and UP, while CN, CSX, and NS saw significant growth due to broad organizational changes. Finally, I develop a metric that determines whether firms substitute inputs towards factors that innovation makes more productive. I estimate the probability that each firm takes that action to be around 50\% with no discernible pattern over time, providing evidence that firms don't anticipate technological change or don't adjust input allocation to take advantage of innovations.
%	\end{blockquote}
%	\emph{Markups and Scale Elasticities for Differentiated Railroad Networks}\\
%	\begin{blockquote}
%	In this paper, we develop and estimate a model that provides both markups and scale elasticities that vary across railroads and through time for the traffic on their networks. Our model is based on a framework provided by Hall (1988) and Klette (1999) wherein markups and scale elasticities are estimated from production relations.  In our model, we aggregate the shipments over each firm's network, which provides a mapping from inputs and network and shipment characteristics to aggregate outputs over the network.  Markups and scale elasticities are taken to follow a multivariate distribution.  This allows for differences in markups and scale across firms and through time, but also for covariances across firms in markups and scale. We estimate the model with Bayesian methods to find markups that are generally well in excess of marginal costs and scale elasticities that generally point to increasing or constant returns in the industry.
%	\end{blockquote}
%	\emph{Competitive Pressures and Inefficiency in Allocation}\\
%	\begin{blockquote}
%	There is a wealth of literature that points to inefficiencies in production. Inefficiencies can arise in the production of outputs from overutilization of inputs in the production process (technical inefficiency) or from errors in optimization that misalign factor prices and optimal input decisions (allocative inefficiency). In examinations of inefficiency, many studies use an inflexible production technology, typically the Cobb-Douglas form, which fails to account for differences in the technology of firms, or fails to control for the effects of competition in limiting inefficiency. In this study, I develop a model that allows for substitutes and complements in production and flexibly accounts for patterns in productivity. I use the model to derive firm cost functions and estimate technical and allocative inefficiencies. Finally, I allow allocative errors to be correlated with the level of competition to examine how the incentive to precisely allocate inputs and minimize costs are affected by competitive pressures.
%	\end{blockquote}
%\end{body}

\end{document}