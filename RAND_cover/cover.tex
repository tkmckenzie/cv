\documentclass[]{article}
\RequirePackage{amsmath}
\RequirePackage{amsfonts}
\usepackage{graphicx}
\usepackage{amssymb}
\usepackage{gensymb}

\setlength{\oddsidemargin}{0in} \setlength{\evensidemargin}{0in}
\setlength{\textwidth}{6.5in} \setlength{\headsep}{0pt}
\setlength{\headheight}{0pt} \setlength{\topmargin}{0in}
\setlength{\textheight}{9.5in}

\begin{document}
\thispagestyle{empty}
%\maketitle

\noindent Taylor K McKenzie\\1017 Adams St SE\\ Albuquerque, NM 87108\\ (509) 430-0031\\ tkmckenzie@gmail.com\\ \\
\today

\subsection*{RAND Corporation}

To Whom It May Concern: \\

I am writing to express strong interest in the Research Economist position at the RAND Corporation. I received my Ph.D. in Economics from the University of Oregon in 2017, pursuing research in industrial organization and applied econometrics. Specifically, my research focused on using flexible structural modeling to describe pricing behavior and estimating different types of inefficiencies in production and planning. The statistical methods used in these analyses were developed to provide novel, robust approaches that addressed precise research questions. I used Bayesian methods to overcome numerical instabilities often encountered with classical methods and permit estimation of more flexible models than have previously been used. I employed and extended a newly developed smooth version of data envelopment analysis to estimate marginal input productivities and allocative inefficiencies. I continue to enjoy analyses that offer the challenge of balancing flexibility and parsimony to arrive at solutions that most effectively address specific research questions.\\

Since graduating, I have been employed at Sandia National Laboratories as a Senior Cybersecurity Researcher. During this time, I have used and improved my knowledge of economic and statistical analyses while also developing skills relevant to cybersecurity. I have gained greater understanding of cyber-attack vectors through formal cybersecurity training and further advanced those abilities through risk and resilience analyses of cyber-dependent systems and red teaming exercises. This research has spanned topics including investigation of low-level cyber vulnerabilities, impact of disruptions to system capabilities and mission, and prioritization of cybersecurity improvements. I have also contributed to economic disaster analyses and built robust statistical models for a variety of applications, learning and developing new methodologies in the process. This experience increased my comprehensive understanding of common econometric issues, which enabled me to apply solutions to applications very different from what I was familiar with. Each of these efforts involved working on teams with diverse backgrounds, drawing on many sources of expertise, and determining how to best synthesize numerous and varied concepts into a coherent framework to be used for robust analyses.\\

I believe my economic and statistical expertise, cybersecurity experience, and familiarity with working among diverse teams will be well-suited to projects at RAND. I look forward to the opportunity to develop novel approaches to address challenging problems and provide objective information to policy-makers, enabling them to make decisions that benefit national security and the public good. I hope to be able to pair my skill set with the extensive abilities of people at RAND and further RAND's reputation for excellence and innovation.\\

I have attached all requested materials with this application, please let me know if there is any other information I can provide. Thank you for your consideration and I look forward to hearing from you.\\\\


\noindent Sincerely,\\

\noindent Taylor McKenzie
\end{document}