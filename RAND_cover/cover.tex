\documentclass[]{article}
\RequirePackage{amsmath}
\RequirePackage{amsfonts}
\usepackage{graphicx}
\usepackage{amssymb}
\usepackage{gensymb}

\setlength{\oddsidemargin}{0in} \setlength{\evensidemargin}{0in}
\setlength{\textwidth}{6.5in} \setlength{\headsep}{0pt}
\setlength{\headheight}{0pt} \setlength{\topmargin}{0in}
\setlength{\textheight}{9.5in}

\begin{document}
\thispagestyle{empty}
%\maketitle

\noindent Taylor K McKenzie\\1017 Adams St SE\\ Albuquerque, NM 87108\\ (509) 430-0031\\ tkmckenzie@gmail.com\\ \\
\today

\subsection*{RAND Corporation}

To Whom It May Concern: \\

I am writing to express strong interest in the research economist position at the RAND Corporation. I received my Ph.D. in Economics from the University of Oregon in 2017, pursuing research in industrial organization and applied econometrics. Specifically, my research focused using flexible structural modeling to describe pricing behavior and estimate different types of inefficiencies in production and planning. The statistical methods used in these analyses were developed to provide novel, robust approaches that addressed specific research questions. Bayesian methods were used to overcome numerical instabilities often encountered with classical methods and permit estimation of more flexible models than have previously been used. A newly developed smooth version of data envelopment analysis was employed and extended to estimate marginal input productivities and allocative inefficiencies. I continue to enjoy analyses that offer the challenge of balancing flexibility and parsimony to arrive at solutions that most effectively address specific research questions.\\

Since graduating, I have been employed at Sandia National Laboratories as a Senior Cybersecurity Researcher. During this time, I have used and improved my knowledge of economic and statistical analyses while also developing skills relevant to cybersecurity. I have gained greater understanding of cyber-attack vectors through formal cybersecurity training and further advanced those abilities through risk and resilience analyses of cyber-dependent systems and red teaming exercises. This research has spanned topics including investigation of low-level cyber vulnerabilities, impact of disruptions to system capabilities and mission, and prioritization of cybersecurity improvements. I have also contributed to economic disaster analyses and built robust statistical models for a variety of applications, learning and developing new methodologies in the process. This experience made me more comprehensively understand common econometric issues so that I could apply solutions to applications very different from what I was familiar with. Each of these efforts involved working on teams with diverse backgrounds, drawing on many sources of expertise, and determining how to best synthesize numerous and varied concepts into a coherent framework to be used for robust analyses.\\

I believe my economic and statistical expertise, cybersecurity experience, and familiarity with working among diverse teams will be well-employed by projects at RAND. I look forward to finding novel solutions to difficult, important, and policy-relevant problems by \\\\=


\noindent Sincerely,\\

\noindent Taylor McKenzie
\end{document}